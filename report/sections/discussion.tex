%!TEX root = ../main.tex

% \par
% This section will largely need to be written after I have results.

% \par
% I imagine that one major takeaway from my project will be the fact that historical and generated data alone don't do a great job of modeling hurricanes.
% I expect to see many outliers, and that the predictions may not be reliable to the degree needed for weather forecasting.
% The incorporation of weather models, and the inclusion of larger scale weather patterns such as El Ni\~{n}o, could likely provide much more accurate results, and would be an interesting direction for future research.

% \par
% I will also mention the fact that larger scale changes in weather patterns, due to global warming, may present some skew in the results.
% It would be helpful to take this into account while trying to fit probability distributions to hurricane data.

\par
In this paper, I've presented a new statistical method for predicting whether or not a hurricane will make landfall.
I use time series similarity metrics to compare hurricanes, and use KPCA to find a representation for them in a Euclidean space.
I then use Nardaraya-Watson kernel regression to approximate the relationship between these embedded coordinates and the probability a hurricane will make landfall.
The analysis section provides evidence supporting the efficacy of my method.

\par
Of course, this is a relatively primitive model, and there's a wealth of available information about a hurricane that I don't utilize.
Factors such as the size of a storm, the extent in various directions of its winds, the barometric pressure, and much more could help tune an extension of the model to be much more accurate.

\par
Another challenge my technique faces is speed.
Computing the similarity between large numbers of hurricanes becomes expensive very quickly.
The number of comparisons is $O(n^{2})$ for the number of hurricanes in the training dataset, and each comparison is $O(\mathscr{L}(A)\mathscr{L}(B))$, even when utilizing dynamic programming.
With more computational resources, it would be possible to train and test on larger datasets, and make full use of the simulation capabilities to generate new data.

\par
Another ineresting area for future research would be high level modeling of hurricane behavior.
This could be useful for further tweaking the simulation work, such as choosing a better mean and/or covariance.

\par
Obviously, only $70\%$ accuracy is not enough to make life-or-death predictions, and any purely statistical model will always be inferior to a more complex model that takes present weather patterns into account.
However, this paper makes it clear that there are merits to using statistical techniques to model and understand hurricanes.