%!TEX root = ../main.tex

\par
I perform two major experiments to analyze the efficacy.
First, I need to determine the underlying dimensionality of the vector space the hurricanes lie in.
To do this, I try embedding a subset of the full dataset into $\R^{i}$ for $i=1,\ldots,15$, and then regressing the Nardaraya-Watson estimator for each case.
I then predict a different subset of the data to evaluate the accuracy of the model, for each number of dimensions.
In Figure ----, I plot the accuracy by the number of dimensions $i$.
In Figure ----, I plot the accuracy (but this time only including points for which the estimator has at least $90\%$ confidence) by the number of dimensions.
Since there's little improvement after $i=15$, I conclude that this is the optimal choice for the number of dimensions.

\begin{figure}
	\centering
	%\includegraphics[width=\linewidth]
	\caption{A plot of accuracy of the estimator versus the number of dimensions the hurricanes are embedded into by KPCA.}
	\label{fig:dimensions}
\end{figure}

\begin{figure}
	\centering
	%\includegraphics[width=\linewidth]
	\caption{A plot of accuracy of the estimator for only points in which it has high confidence (at least $90\%$) versus the number of dimensions the hurricanes are embedded into by KPCA.}
	\label{fig:confident_dimensions}
\end{figure}