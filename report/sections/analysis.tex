%!TEX root = ../main.tex

\par
I perform two major experiments to analyze the efficacy.
First, I need to determine the underlying dimensionality of the vector space the hurricanes lie in.
To do this, I try embedding a subset of the full dataset into $\R^{i}$ for $i=1,\ldots,15$, and then regressing the Nardaraya-Watson estimator for each case.
I then predict a different subset of the data to evaluate the accuracy of the model, for each number of dimensions.
In Figure~\ref{fig:dimensions}, I plot the accuracy by the number of dimensions $i$.
% In Figure~\ref{fig:confident_dimensions}, I plot the accuracy (but this time only including points for which the estimator has at least $90\%$ confidence) by the number of dimensions.
% Generally, there's improvement in both accuracies up to $i=15$, but not after, so I conclude that this is the optimal choice for the number of dimensions.
Generally, the accuracy is roughly improving up to $i=15$, but not after, so I conclude that this is the optimal choice for the number of dimensions.

\par
Now that I have fixed a dimension, I can test on a larger data set.
In order to fully evaluate how accurate my algorithm is, I train it on every huurricane from 2000 through 2018, and then test it on every hurricane from 1970 through 1999.
Out of $479$ hurricanes in the test dataset, the learned model correctly predicts $335$.
This is an accuracy of $0.70$.