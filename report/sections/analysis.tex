%!TEX root = ../main.tex

\par
I perform two major experiments to analyze the efficacy.
First, I need to determine the underlying dimensionality of the vector space the hurricanes lie in.
To do this, I try embedding a subset of the full dataset into $\R^{i}$ for $i=1,\ldots,15$, and then regressing the Nardaraya-Watson estimator for each case.
I then predict a different subset of the data to evaluate the accuracy of the model, for each number of dimensions.
In Figure~\ref{fig:dimensions}, I plot the accuracy by the number of dimensions $i$.
In Figure~\ref{fig:confident_dimensions}, I plot the accuracy (but this time only including points for which the estimator has at least $90\%$ confidence) by the number of dimensions.
The overall accuracy is roughly improving up to $i=15$, but not after.
On the other hand, the accuracy for high confidence points appears to be optimal at $i=10$.
This makes it difficult to judge the underlying dimension of the data, so I will experiment with both choices of dimension.

\par
First, I will test with $i=10$.
In order to fully evaluate how accurate my algorithm is, I train it on every huurricane from 2000 through 2018, and then test it on every hurricane from 1970 through 1999.
Out of --- hurricanes in the test dataset, the learned model correctly predicts ---; this is an accuracy of ----.

\par
Now, I will test with $i=15$.
In order to fully evaluate how accurate my algorithm is, I train it on every huurricane from 2000 through 2018, and then test it on every hurricane from 1970 through 1999.
Out of $479$ hurricanes in the test dataset, the learned model correctly predicts $335$; this is an accuracy of $0.70$.

\par
Additionally, in the case where $i=15$, I plot all of the successfully predicted storms in Figure -----, and all of the unsuccessfully predicted storms in Figure ------.
There's a clear trend -- storms further to the north are significantly easier to predict in the early stages of their lifetime just using statistics, whereas storms further to the south are more challenging.