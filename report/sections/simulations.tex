%!TEX root = ../main.tex

\par
In order to generate more data from the limited real-world data available, I can add Gaussian noise to existing data.
One way to do so is simply adding Gaussian noise to each point in a track.
A more complex way is generating random walks of the same length, and then add them pointwise to the track.
I will analyze how realistic the data generated by these methods are, and use the larger dataset to examine the quality of my estimators.

\par
The first way to add noise to a track is to just independently add Gaussian noise to each point.
Let $A=[(\vec{x}_{1}),\ldots,(\vec{x}_{n})]$ (with $\vec{x}_{i}\in\R^{2}$) be the track of a hurricane.
For $i=1,\ldots,n$, let $\varepsilon_{i}\sim\mathcal{N}(\mu,\Sigma)$, where $\mu\in\R^{2}$ and $\Sigma\in\R^{2\times{}2}$ are some mean and covariance.
Then our new hurricane track is
\[
	A'=[\vec{x}_{1}+\varepsilon_{1},\ldots,\vec{x}_{n}+\varepsilon_{n}]
\]

\par
A more advanaced way to add noise to a track is to compute a Gaussian random walk of the same length as the track, and then add them together.
Let $A$ be the track of the hurricane as before.
For $i=1,\ldots,n$, let $\varepsilon_{i}\sim\mathcal{N}(\vec{x}_{i-1}+\varepsilon_{i-1},\Sigma)$ and $\varepsilon_{0}\sim\mathcal{N}(\mu,\Sigma)$ for some initial mean $\mu$ and covariance $\Sigma$.
Once again, our new hurricane track is
\[
	A'=[\vec{x}_{1}+\varepsilon_{1},\ldots,\vec{x}_{n}+\varepsilon_{n}]
\]